%%%%%%%%%%%%%%%%%%%%%
\chapter*{Apêndices}%
%%%%%%%%%%%%%%%%%%%%%


%%%%%%%%%%%%%%%%%%%%%%%%%%%%%%%%%%%%%%%%%%%%%%%%%%%%%%%%%%%%%%%%%%%%%%%%%%%%%%%%%%%%%%%
\section*{Apêndice A - Dedução do Modelo Recursivo da Média no Tempo}\label{apendiceA}%
%%%%%%%%%%%%%%%%%%%%%%%%%%%%%%%%%%%%%%%%%%%%%%%%%%%%%%%%%%%%%%%%%%%%%%%%%%%%%%%%%%%%%%%

Partindo-se da equação da média no tempo \eqref{y}, e fazendo-se um avanço de uma iteração no vetor da média, tem-se:

\begin{equation}\nonumber
y(k) = \frac{1}{k+1} \sum^{k}_{l=0} x(l)
\end{equation}

\begin{equation}\nonumber
y(k+1) = \frac{1}{k+2} \sum^{k+1}_{l=0} x(l)
\end{equation}

\begin{equation}\nonumber
y(k+1) = \frac{1}{k+2} [x(0) + x(1) + ... + x(k) + x(k+1)]
\end{equation}

\begin{equation}\nonumber
y(k+1) = \frac{1}{k+2} \sum^{k}_{l=0} x(l) + \frac{1}{k+2} x(k+1)
\end{equation}

\begin{equation}\nonumber
y(k+1) = \frac{1}{k+2} (k+1)y(k) + \frac{1}{k+2} x(k+1)
\end{equation}\\

Por fim, o método recursivo da média no tempo é dado pela seguinte equação:

\begin{equation}\nonumber
y(k+1) = \frac{k+1}{k+2} y(k) + \frac{1}{k+2} x(k+1)
\end{equation}\\

Esta equação segue o mesmo formato da apresentada em \eqref{yr}.


%%%%%%%%%%%%%%%%%%%%%%%%%%%%%%%%%%%%%%%%%%%%%%%%%%%%%%%%%%%%%%%%%%%%%%%%%%%%%%%%%%%%%%%%%
\section*{Apêndice B - Código em Linguagem MATLAB Usado nas Simulações}\label{apendiceB}%
%%%%%%%%%%%%%%%%%%%%%%%%%%%%%%%%%%%%%%%%%%%%%%%%%%%%%%%%%%%%%%%%%%%%%%%%%%%%%%%%%%%%%%%%%

\lstinputlisting{apendice/pagerank.m}